\chapter*{\large ВВЕДЕНИЕ}  
\addcontentsline{toc}{chapter}{ВВЕДЕНИЕ}
В современном мире очень много разных подходов к моделированию и анализу физических процессов. Главной задачей всегда является вычисление и отслеживание изменений некоторых параметров модели и ее объектов. Но, когда объект представляет собой сложную фигуру, чаще всего посчитать какой-то параметр довольно сложно. Приходится находить методы для упрощения вычислений.

Расчетные сетки являются одним из методом анализа объектов. Они представльют собой разбиение на маленькие части, которые, как правило, проще анализировать. Расчетные сетки либо заполняют, либо покрывают объект, и таким образом вычисления параметров по рассчетным сеткам будут аппроксимациями вычислений параметров самого объекта.  

В данной работе будут изучены программные возможности для генерации, визуализации и работы с сетками. Будут построены сетки, для нескольких фигур в качестве примеров. Будут разработаны и реализованы алгоритмы для расчета центра масс и сферичности, объема, ограниченного поверхностью, на основании анализа двух несогласованных сеток (сетка для объема и сетка для поверхности).