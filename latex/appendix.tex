\titleformat{\section}[block]
  {\large\bfseries\centering}
  {\thesection\ }{}{}
\chapter*{ПРИЛОЖЕНИЕ А}
\addcontentsline{toc}{chapter}{ПРИЛОЖЕНИЕ А}
\section*{\centering Код программы. Функции для вычисления параметров }
\begin{footnotesize}
\begin{lstlisting}
# volume
def volume(a,b,c,d):
    import numpy as np
    ab = np.array(b) - np.array(a)
    ac = np.array(c) - np.array(a)
    ad = np.array(d) - np.array(a)
    return 1/6 *np.linalg.det(np.matrix([ab,ac,ad]))


# center of mass part for one tetrahedron
def centre_of_mass(a,b,c,d, i):
    return 0.25*(a[i]+b[i]+c[i]+d[i])*volume(a,b,c,d)


# area of triangle   
def area_of_triangle(a,b,c):
    import numpy as np
    ab = np.array(b) - np.array(a)
    ac = np.array(c) - np.array(a)
    return 0.5*np.dot(ab,ac)

# area of quadrilateral
def area_of_quadrelateral(a,b,c,d):
    return area_of_triangle(a,b,c) + area_of_triangle(a,c,d)


# sphericity
def sphericity(area, volume):
    import math
    return (math.pi**(1/3)) *((6*volume)**(2/3))/area
\end{lstlisting}
\end{footnotesize}
\chapter*{ПРИЛОЖЕНИЕ Б}
\addcontentsline{toc}{chapter}{ПРИЛОЖЕНИЕ Б}
\section*{\centering Код программы.Построение и анализ сферы}
\begin{footnotesize}
\begin{lstlisting}
def main():
    import numpy
    from meshpy.tet import MeshInfo, build, Options
    from meshpy.geometry import GeometryBuilder, Marker,make_box,make_ball
    import volume
    import time
    start_time = time.time()
    # mesh generation
    geom = GeometryBuilder();
    box_marker = Marker.FIRST_USER_MARKER
    points, facets, _, _ = make_ball(r = 0.25)
    for i in range(len(points)):
        a = [0,0,0]
        a = numpy.array(points[i]) + numpy.array([0.5,0.5,0.5])
        points[i] = tuple(a)
    area = 0
    # surface area count using facets
    for i in range(len(facets)):
        if(len(facets[i][0]) == 4):   
            [[p1,p2,p3,p4]] = facets[i]
            [pt1,pt2,pt3,pt4] = [points[p1],points[p2],points[p3],points[p4]]
            area += volume.area_of_quadrelateral(pt1,pt2,pt3,pt4)
        elif(len(facets[i][0]) == 3):
            [[p1,p2,p3]] = facets[i]
            [pt1,pt2,pt3] = [points[p1],points[p2],points[p3]]
            area += volume.area_of_triangle(pt1,pt2,pt3)
    print("area: " + str(area))
    geom.add_geometry(points, facets, facet_markers=box_marker)
    mesh_info = MeshInfo()
    geom.set(mesh_info)
    mesh = build(mesh_info, max_volume=0.01,
                 volume_constraints=True, attributes=True)
    mesh.element_volumes.setup()
    tot = 0;
    #volume count
    for i in range(len(mesh.elements)):
        [p1,p2,p3,p4] = mesh.elements[i]
        [pt1,pt2,pt3,pt4] = [mesh.points[p1], mesh.points[p2],
        mesh.points[p3], mesh.points[p4]]
        tot += volume.volume(pt1,pt2,pt3,pt4)
    print("volume: " +  str(tot))
    d = [0,0,0]
    for i in range(len(mesh.elements)):
        [p1,p2,p3,p4] = mesh.elements[i]
        [pt1,pt2,pt3,pt4] = [mesh.points[p1], mesh.points[p2],
        mesh.points[p3], mesh.points[p4]]
        for i in range(3):
            d[i]+= volume.centre_of_mass(pt1,pt2,pt3,pt4,i)
    #center of mass, sphericity and time count
    print("center of mass: " + str(d/tot))
    print("sphericity: " + str(volume.sphericity(area,tot)))
    print("timing: %s seconds" % (time.time() - start_time))
    #export to vtk file
    mesh.write_vtk("G0.vtk")
    

if __name__ == "__main__":
    main()
\end{lstlisting}
\end{footnotesize}

\chapter*{ПРИЛОЖЕНИЕ В}
\addcontentsline{toc}{chapter}{ПРИЛОЖЕНИЕ В}
\section*{\centering Код программы.Построение и анализ последовательности эллипсоидов}
\begin{footnotesize}
\begin{lstlisting}
def analysis(radius, scalar, center, n):
    import numpy
    from meshpy.tet import MeshInfo, build, Options
    from meshpy.geometry import GeometryBuilder, Marker, make_box, make_ball
    import volume
    geom = GeometryBuilder()
    box_marker = Marker.FIRST_USER_MARKER
    points, facets, _, _ = make_ball(r=radius)
    for i in range(len(points)):
        a = numpy.array(points[i])
        a[2] *= scalar
        a += numpy.array(center)
        points[i] = tuple(a)
    area = 0
    for i in range(len(facets)):
        if len(facets[i][0]) == 4:
            [[p1, p2, p3, p4]] = facets[i]
            [pt1, pt2, pt3, pt4] = [points[p1], points[p2],
            points[p3], points[p4]]
            area += volume.area_of_quadrelateral(pt1, pt2, pt3, pt4)
        elif len(facets[i][0]) == 3:
            [[p1, p2, p3]] = facets[i]
            [pt1, pt2, pt3] = [points[p1], points[p2], points[p3]]
            area += volume.area_of_triangle(pt1, pt2, pt3)
    print("area: " + str(area))
    geom.add_geometry(points, facets, facet_markers=box_marker)
    mesh_info = MeshInfo()
    geom.set(mesh_info)
    mesh = build(mesh_info, max_volume=0.01,
                 volume_constraints=True, attributes=True)
    mesh.element_volumes.setup()
    # volume count
    tot = 0
    for i in range(len(mesh.elements)):
        [p1, p2, p3, p4] = mesh.elements[i]
        [pt1, pt2, pt3, pt4] = [mesh.points[p1], mesh.points[p2],
        mesh.points[p3], mesh.points[p4]]
        tot += volume.volume(pt1, pt2, pt3, pt4)
    print("volume: " + str(tot))
    #center of mass count
    d = [0, 0, 0]
    for i in range(len(mesh.elements)):
        [p1, p2, p3, p4] = mesh.elements[i]
        [pt1, pt2, pt3, pt4] = [mesh.points[p1], mesh.points[p2],
        mesh.points[p3], mesh.points[p4]]
        for j in range(3):
            d[j] += volume.centre_of_mass(pt1, pt2, pt3, pt4, j)

    print("center of mass: " + str(d / tot))
    #sphericity count
    print("spherical: " + str(volume.sphericity(area, tot)))
    # file write
    # mesh.write_vtk("G"+str(n)+".vtk")


def main():
    a = []
    for i in range(10):
        a.append(1/(i+1))
    for i in range(len(a)):
        print("ellipse"+str(i+1))
        analysis(1,a[i],[0,0,0], i+1)
        print("====================")
        

if __name__ == "__main__":
    main()

\end{lstlisting}
\end{footnotesize}

\chapter*{ПРИЛОЖЕНИЕ Г}
\addcontentsline{toc}{chapter}{ПРИЛОЖЕНИЕ Г}
\section*{\centering Код программы.Построение и анализ сетки методом выбора тетраэдров}
\begin{footnotesize}
\begin{lstlisting}
# surface
def function(x,y,z):
    return (x-0.5)**2 + (y-0.5)**2 + (z-0.5)**2 - 1/16


def main():
    import numpy
    from meshpy.tet import MeshInfo, build, Options
    from meshpy.geometry import GeometryBuilder, Marker, make_box, make_ball
    import volume
    import math
    import time
    # mesh generation
    start_time = time.time()
    geom = GeometryBuilder();
    box_marker = Marker.FIRST_USER_MARKER
    points, facets, _, facet_markers = make_box(
        numpy.array([0, 0, 0]), numpy.array([1, 1, 2])
    )
    geom.add_geometry(points, facets, facet_markers=box_marker)
    mesh_info = MeshInfo()
    geom.set(mesh_info)
    mesh_info.regions.resize(1)
    mesh_info.regions[0] = ([0, 0, 0] + [1, 0.001, ])
    mesh = build(mesh_info, max_volume=0.01,
                 volume_constraints=True, attributes=True)

    in_region = []
    for i in range(len(mesh.elements)):
        [p1, p2, p3, p4] = mesh.elements[i]
        [pt1, pt2, pt3, pt4] = [mesh.points[p1], mesh.points[p2],
        mesh.points[p3], mesh.points[p4]]
        # selection of inside tetrahedra
        # if (function(pt1[0],pt1[1],pt1[2]) <= 0
        #         and function(pt2[0],pt2[1],pt2[2]) <= 0 and
        function(pt3[0],pt3[1],pt3[2]) <= 0
        #         and function(pt4[0],pt4[1],pt4[2] <= 0)):
        #     in_region.append(i)
        # selection of inside and intersecting tetrahedra
        if (function(pt1[0], pt1[1], pt1[2]) <= 0
                or function(pt2[0], pt2[1], pt2[2]) <= 0 or function(pt3[0],
                pt3[1], pt3[2]) <= 0
                or function(pt4[0], pt4[1], pt4[2] <= 0)):
            in_region.append(i)
    print(str(len(in_region)/len(mesh.elements)))
    tot = 0
    # volume
    for i in range(len(in_region)):
        [p1, p2, p3, p4] = mesh.elements[in_region[i]]
        [pt1, pt2, pt3, pt4] = [mesh.points[p1], mesh.points[p2],
        mesh.points[p3], mesh.points[p4]]
        tot += volume.volume(pt1, pt2, pt3, pt4)
    print("volume: " + str(tot))
    # center of mass
    d = [0, 0, 0]
    for i in range(len(in_region)):
        [p1, p2, p3, p4] = mesh.elements[in_region[i]]
        [pt1, pt2, pt3, pt4] = [mesh.points[p1], mesh.points[p2],
        mesh.points[p3], mesh.points[p4]]
        for j in range(3):
            d[j] += volume.centre_of_mass(pt1, pt2, pt3, pt4, j)
    print("center of mass: " + str(d / tot))
    # surface area for sphericity is counted by formula for a sphere
    print("sphericity: " + str(volume.sphericity(math.pi*0.25,tot)))
    print("timing: %s seconds" % (time.time() - start_time))
    

if __name__ == "__main__":
    main()

\end{lstlisting}
\end{footnotesize}

\chapter*{ПРИЛОЖЕНИЕ Д}
\addcontentsline{toc}{chapter}{ПРИЛОЖЕНИЕ Д}
\section*{\centering Код программы.Функции для нахождения точек пересечения}
\begin{footnotesize}
\begin{lstlisting}
#intersection test
def intersectQ(p1,p2,surface):
    a = surface(p1[0], p1[1],p1[2])
    b = surface(p2[0], p2[1], p2[2])
    if a * b == 0:
        return 0
    return a * b / abs(a * b)

# finding intersection points
def intersection_point(p1,p2,surface):
    from sympy import symbols, solve
    import numpy as np
    t = symbols('t')
    x = (p2[0] - p1[0]) * t + p1[0]
    y = (p2[1] - p1[1]) * t + p1[1]
    z = (p2[2] - p1[2]) * t + p1[2]
    sol = solve(surface(x,y,z))
    for i in sol:
        x = (p2[0] - p1[0]) * i + p1[0]
        y = (p2[1] - p1[1]) * i + p1[1]
        z = (p2[2] - p1[2]) * i + p1[2]
        if np.dot(np.array(p2) - np.array([x,y,z]),
        np.array([x,y,z]) - np.array(p1)) > 0:
            return [x,y,z]
        
# surface 
def function(x,y,z):
    return (x-0.5)**2 + (y-0.5)**2 + (z-0.5)**2 - 1/16
    
# example
def main():
    from intersection import tetrahedral_division
    p1 = [0, 0, 0]
    p2 = [2, 0, 0]
    p3 = [0, 2, 0]
    p4 = [0, 0, 2]
    surface = function
    sides = tetrahedral_division(p1, p2, p3, p4)
    a = []
    for i in sides:
        if intersectQ(i[0], i[1], surface) == -1:
            print(intersection_point(i[0],i[1],surface))

if __name__ == "__main__":
    main()


\end{lstlisting}
\end{footnotesize}

\chapter*{ПРИЛОЖЕНИЕ Е}
\addcontentsline{toc}{chapter}{ПРИЛОЖЕНИЕ Е}
\section*{\centering Код программы.Постоение и анализ сетки методом выпуклой оболочки}
\begin{footnotesize}
\begin{lstlisting}
 def main():
    import numpy
    from meshpy.tet import MeshInfo, build, Options
    from meshpy.geometry import GeometryBuilder, Marker, make_box
    import volume
    import math
    from scipy.spatial import ConvexHull
    import surface_intersection as si
    from intersection import tetrahedral_division
    import sys
    import time
    #mesh generation
    geom = GeometryBuilder()
    box_marker = Marker.FIRST_USER_MARKER
    points, facets, _, facet_markers = make_box(
        numpy.array([0, 0, 0]), numpy.array([1, 1, 2])
    )
    geom.add_geometry(points, facets, facet_markers=box_marker)
    mesh_info = MeshInfo()
    geom.set(mesh_info)
    mesh_info.regions.resize(1)
    mesh_info.regions[0] = ([0, 0, 0] + [1, 0.001, ])
    mesh = build(mesh_info, max_volume=0.01,
                 volume_constraints=True, attributes=True)
    #finding intersection points
    surface = si.function
    in_region = []
    in_region_tetr = []
    total = len(mesh.elements)
    start_time = time.time()
    for i in range(total):
        [p1, p2, p3, p4] = mesh.elements[i]
        [pt1, pt2, pt3, pt4] = [mesh.points[p1], mesh.points[p2],
        mesh.points[p3], mesh.points[p4]]
        sides = tetrahedral_division(pt1, pt2, pt3, pt4)
        print(str(int(100*i/total)) + '%')
        sys.stdout.write('\x1b[1A')
        sys.stdout.write('\x1b[2K')
        for j in sides:
            if si.intersectQ(j[0], j[1], surface) == -1:
                in_region.append(si.intersection_point(j[0],j[1],surface))
                in_region_tetr.append(i)
    print("timing: %s seconds" % (time.time() - start_time))
    #counting number of intersected tetrahedrons
    print(str(len(in_region))+"/"+ str(len(mesh.elements)))
    hull = ConvexHull(numpy.array(in_region))
    # volume, area and sphericity
    print("volume: " + str(hull.volume))
    print("area: " + str(hull.area))
    print("sphericity: " + str(volume.sphericity(hull.area,hull.volume)))
    tot = 0
    d = [0, 0, 0]
    for i in range(len(in_region_tetr)):
        [p1, p2, p3, p4] = mesh.elements[in_region_tetr[i]]
        [pt1, pt2, pt3, pt4] = [mesh.points[p1], mesh.points[p2],
        mesh.points[p3], mesh.points[p4]]
        for j in range(3):
            d[j] += volume.centre_of_mass(pt1, pt2, pt3, pt4, j)
        tot+= volume.volume(pt1,pt2,pt3,pt4)
    #center of mass
    print("center of mass: " + str(d/tot))
    print("timing: %s seconds" % (time.time() - start_time))


if __name__ == "__main__":
    main()

\end{lstlisting}
\end{footnotesize}
